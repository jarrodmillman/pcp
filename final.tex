\documentclass{article}
\usepackage[T1]{fontenc}
\usepackage{amssymb, amsmath, graphicx, subfigure}
\usepackage{diagbox}
\usepackage{hyperref}
\usepackage{enumitem}
\usepackage[round]{natbib}

% Commands for annotating the docs with fixme and inter-author notes.  See
% below for how to disable these.
%
% Define a \fixme command to mark visually things needing fixing in the draft,
% as well as similar commands for each author to leave initialed special
% comments in the document.
% For final printing or to simply disable these bright warnings, copy
% (there's a target macros_off' in the makefile that does this) the file
% macros_off.tex to macros.tex
\newcommand{\fixme}[1] { \textcolor{red} {
{\fbox{ {\bf Fix:} \ensuremath{\blacktriangleright }} {\bf #1}
\fbox{\ensuremath{\blacktriangleleft} } } } }



\setlength{\oddsidemargin}{.25in}
\setlength{\evensidemargin}{.25in}
\setlength{\textwidth}{6in}
\setlength{\topmargin}{-0.4in}
\setlength{\textheight}{8.5in}

\newcommand{\heading}[6]{
  \renewcommand{\thepage}{\arabic{page}}
  \noindent
  \begin{center}
  \framebox{
    \vbox{
      \hbox to 5.78in { \textbf{#2} \hfill #3 }
      \vspace{4mm}
      \hbox to 5.78in { {\Large \hfill #6  \hfill} }
      \vspace{2mm}
      \hbox to 5.78in { {\large \hfill ``On Dinur's Proof of the PCP Theorem'' \hfill} }
      \vspace{3mm}
      \hbox to 5.78in { \textit{Instructor: #4 \hfill #5} }
    }
  }
  \end{center}
  \vspace*{4mm}
}

\newtheorem{theorem}{Theorem}
\newtheorem{definition}[theorem]{Definition}
\newtheorem{remark}[theorem]{Remark}
\newtheorem{lemma}[theorem]{Lemma}
\newtheorem{corollary}[theorem]{Corollary}
\newtheorem{proposition}[theorem]{Proposition}
\newtheorem{claim}[theorem]{Claim}
\newtheorem{observation}[theorem]{Observation}
\newtheorem{fact}[theorem]{Fact}
\newtheorem{assumption}[theorem]{Assumption}

\newenvironment{proof}{\noindent{\bf Proof:} \hspace*{1mm}}{
	\hspace*{\fill} $\Box$ }
\newenvironment{proof_of}[1]{\noindent {\bf Proof of #1:}
	\hspace*{1mm}}{\hspace*{\fill} $\Box$ }
\newenvironment{proof_claim}{\begin{quotation} \noindent Proof: \hspace*{1mm}}{
	\hspace*{\fill} $\diamond$ \end{quotation}}

\newenvironment{algorithm}{\noindent{\bf Algorithm:} \hspace*{1mm}}{
	\hspace*{\fill} $\Box$ }
\newenvironment{strategy}{\noindent{\bf Strategy:} \hspace*{1mm}}{
	\hspace*{\fill} $\Box$ }

\newcommand{\problemset}[3]{\heading{}{CS276: Cryptography}{#2}{Alessandro Chiesa}{#3}{Final Project}}


%%%%%%%%%%%%%%%%%%%%%%%%%%%%%%%%%%%%%%%%%%%%%%%%%%%%%%%%%%%%%%%%%%%%%%%%%%%%%%%
% PLEASE MODIFY THESE FIELDS AS APPROPRIATE
\newcommand{\problemsetnum}{4}          % problem set number
\newcommand{\duedate}{November 30, 2017}  % problem set deadline
\newcommand{\studentname}{K. Jarrod Millman}    % full name of student (i.e., you)
% PUT HERE ANY PACKAGES, MACROS, etc., ADDED BY YOU

%-----------------------------------------------------------------------------
% Special-purpose color definitions (dark enough to print OK in black and white)
\usepackage{color}

% A few colors to replace the defaults for certain link types
\definecolor{orange}{cmyk}{0,0.4,0.8,0.2}
\definecolor{darkorange}{rgb}{.71,0.21,0.01}
\definecolor{darkgreen}{rgb}{.12,.54,.11}

%-----------------------------------------------------------------------------
% The hyperref package gives us a pdf with properly built
% internal navigation ('pdf bookmarks' for the table of contents,
% internal cross-reference links, web links for URLs, etc.)
\usepackage{hyperref}

\hypersetup{pdftex,  % needed for pdflatex
  breaklinks=true,  % so long urls are correctly broken across lines
  colorlinks=true,
  urlcolor=blue,
  linkcolor=darkorange,
  citecolor=darkgreen,
  }

%%%%%%%%%%%%%%%%%%%%%%% general useful macros
\usepackage{mathtools}
\newcommand{\N}{{\mathbf N}}
\newcommand{\Z}{{\mathbf Z}}
\newcommand{\F}{{\mathbf F}}
\newcommand{\Q}{{\mathbf Q}}
\newcommand{\Time}{\operatorname{time}}
\newcommand{\poly}{{\mathrm{poly}}}
\newcommand{\polylog}{{\mathrm{polylog}}}
\newcommand{\loglog}{{\mathop{\mathrm{loglog}}}}
\newcommand{\E}{\operatorname{E}}
\DeclarePairedDelimiter{\floor}{\lfloor}{\rfloor}
\DeclarePairedDelimiter{\ceil}{\lceil}{\rceil}
\newcommand{\comp}[1]{\overline{#1}}



\newcommand{\bits}{\{0,1\}}
\newcommand{\class}[1]{\mathbf{#1}}
\newcommand{\coclass}[1]{\mathbf{co\mbox{-}#1}} % and their complements
\newcommand{\BPP}{\class{BPP}}
\newcommand{\NP}{\class{NP}}
\newcommand{\PCP}{\class{PCP}}
\newcommand{\coNP}{\coclass{NP}}
\newcommand{\RP}{\class{RP}}
\newcommand{\coRP}{\coclass{RP}}
\newcommand{\ZPP}{\class{ZPP}}
\newcommand{\RNC}{\class{RNC}}
\newcommand{\RL}{\class{RL}}
\renewcommand{\L}{\class{L}}
\newcommand{\coRL}{\coclass{RL}}
\newcommand{\IP}{\class{IP}}
\newcommand{\AM}{\class{AM}}
\newcommand{\MA}{\class{MA}}
\renewcommand{\P}{\class{P}}
\newcommand\prBPP{\class{prBPP}}
\newcommand\prRP{\class{prRP}}
\newcommand\prP{\class{prP}}
\newcommand{\Ppoly}{\class{P/poly}}
\newcommand{\DTIME}{\class{DTIME}}
\newcommand{\ETIME}{\class{E}}
\newcommand{\BPTIME}{\class{BPTIME}}
\newcommand{\EXP}{\class{EXP}}
\newcommand{\SUBEXP}{\class{SUBEXP}}
\newcommand{\qP}{\class{\tilde{P}}}
\newcommand{\PH}{\class{PH}}
\newcommand{\NC}{\class{NC}}
\newcommand{\PSPACE}{\class{PSPACE}}
\newcommand{\quasiP}{\class{\tilde{P}}}
\newcommand{\BPAC}{\class{BPAC_0}}
\newcommand{\qAC}{\class{\widetilde{AC}_0}}

\newcommand{\PRIMES}{\mathsf{PRIMES}}
\newcommand{\GEN}{\mathsf{GEN}}

\newcommand{\negl}{{\mathrm{neg}}}
\newcommand{\Diam}{\mathrm{Diam}}
\newcommand{\Cut}{\mathrm{Cut}}
\newcommand{\pf}{\mathit{pf}}
\newcommand{\Col}{\mathrm{Col}}
\newcommand{\Supp}{\mathrm{Supp}}
\newcommand{\iid}{\stackrel{\mathrm{iid}}{\sim}}
\newcommand{\eqdef}{\mathbin{\stackrel{\rm def}{=}}}

%%%%%%%%%%%%%%%%%%%%%%%%%%%%%%%%%%%%%%%%%%%%%%%%%%%%%%%%%%%%%%%%%%%%%%%%%%%%%%%


%%%%%%%%%%%%%%%%%%%%%%%%%%%%%%%%%%%%%%%%%%%%%%%%%%%%%%%%%%%%%%%%%%%%%%%%%%%%%%%
\begin{document}
\problemset{\problemsetnum}{\duedate}{\studentname}

%%%%%%%%%%%%%%%%%%%%%%%%%%%%%%%%%%%%%%%%%%%%%%%%%%%%%%%%%%%%%%%%%%%%%%%%%%%%%%%
\abstract{\cite{radhakrishnan2007dinur} discuss the proof by
\cite{dinur2007pcp} of the PCP theorem.}

\section{Introduction}

In class, we discussed interactive proof systems introduced by
\citet*{goldwasser1989knowledge}.
In particular, we saw we could view $\NP$ as a simple proof system with a
deterministic polynomial-time verifier interacting with an all-powerful prover.
We also saw that by providing the verifier access to randomness and assuming
the existence of one-way functions, we could obtain systems with the remarkable
property of zero-knowledge.

For the final project, I learned about a related line of work involving
probabilistic proof systems.
%another line of thought developed from
%this concept of computation through interaction. 
%This line of thought begins by considering a probabilistic verifier with oracle
%access to the proof.
\citet*{feige1996interactive} showed that probabilistic proof systems
were closely related to hardness of approximations.
Building on this relationship, \citet*{arora1998probabilistic} and
\citet*{arora1998proof} proved the PCP Theorem, which shows that a
probabilistic verifier with logarithmic randomness and constant number of
queries provides a new way of characterizing $\NP$.
The original proof was algebraic and difficult.  \cite{dinur2007pcp} provided a
relatively simple, new proof with a combinatorial flavor.

%In section~\ref{pcp}, I briefly recall the notion of an interactive proof
%system.
%Then I define probabilistically checkable proofs and formally state the PCP
%theorem.
%Next I introduce constraints satisfaction problems (CSPs) and the CSP variant
%of the PCP theorem.
%I conclude the section with a proof of the equivalence of these two theorems.
%Finally, in section~\ref{proof}, I briefly outline Dinur's proof of the PCP
%theorem.

%\section{Preliminaries}

\section{PCP Theorem}\label{pcp}

A \emph{system of logic} (or a proof system) is composed of (1) axioms
and (2) rules of derivation.
A proof of a claim consists of a sequence of sentences ending with the claim
such that each sentence is an axiom or may be derived according to the rules
from the previous sentences in the proof.
We desire two main properties from such a system:
\begin{itemize}[leftmargin=10em]
\item[\textbf{(Completeness)}] True claims have a proof.
\item[\textbf{(Soundness)}] False claims have no proof.
\end{itemize}

%Experience suggests that there is a difference in difficulty between
%Experience suggests that producing a proof and verifying a proof
%are distinct activities.
%Using the verifier-based definition of $\NP$, it is easy to see that
%we can characterize $\NP$ as  
%
%Interactive proof systems involve the interactions between
%two parties: an all-powerful prover $P$ and an efficient
%verifier $V$.
%The idea is that for an claim $x$, the prover $P$ provides a
%proof $\pi$ of the claim $x$ to the verifier $V$.
%More formally,\\
%\begin{definition}
Computationally, a \emph{traditional proof system} for a language $L \subseteq \bits^*$
is a deterministic, polynomial-time verifier $V$ with these properties:
\begin{itemize}[leftmargin=10em]
\item[\textbf{(Completeness)}] $\;x \in\; L \;\implies\; \exists \pi \quad V(x, \pi) \text{ accepts}$.
\item[\textbf{(Soundness)}] $\;x \notin\; L \;\implies\; \forall \pi \quad V(x, \pi) \text{ rejects}$.
\end{itemize}
%\end{definition}
You can imagine the verifier $V$ checking that each sentence of the proof $\pi$
is an axiom or derived from previous sentences in the proof $\pi$ to see how
the computational perspective corresponds to a system of logic.

Since $V$ runs in polynomial time in $|x|$, we must have
$|\pi| \le p(|x|)$ for some polynomial $p$.
It is clear---from the verifier-based definition of $\NP$---that traditional proof
systems are an equivalent characterization of $\NP$.
So we may view the theory of $\NP$-completeness as providing a collection of
equivalent traditional proof systems. 

...

...

...

...

...

...


\subsection{Probabilistically Checkable Proofs}

\begin{definition}[$(r(n), q(n))$-restricted verifier]
A probabilistic polynomial time verifier $V$ is \emph{$(r(n), q(n))$-restricted} if,
for all input $x$ and every proof $\pi$, $V^\pi(x)$ makes at most $q(n)$ oracle queries
to $\pi$ and uses at most $r(n)$ random bits.
\end{definition}


\begin{definition}[PCP complexity classes]
We say a language $L$ is in $\PCP[r(n), q(n)]$ if there is a
$(r(n), q(n))$-restricted verifier $V$ such that
\begin{itemize}[leftmargin=10em]
\item[\textbf{(Completeness)}] $\;x \in\; L \;\implies\; \exists \pi \quad \Pr[V^\pi(x) \text{ accepts}] = 1$.
\item[\textbf{(Soundness)}] $\;x \notin\; L \;\implies\; \forall \pi \quad \Pr[V^\pi(x) \text{ accepts}] \le 1/2$.
\end{itemize}
\end{definition}

\begin{theorem}[PCP Theorem]
$$
\NP = \PCP[O(\log n), O(1)]
$$
\end{theorem}

\subsection{Constraint Satisfaction Problems}

\begin{definition}[CSP]
\end{definition}

\begin{theorem}[CSP variant of the PCP Theorem]
\end{theorem}

\subsection{Equivalence of the two notions}

\section{Overview of Dinur's proof}\label{proof}

take a result with a small soundness gap $1 - 1/m$
gradually amplify the soundness to get the
desired gap while keeping the length of the proof constant.

uses the CSP variant of the PCP theorem:

\subsection{Expanders}

\bibliographystyle{plainnat}
\bibliography{final}

\end{document}
%%%%%%%%%%%%%%%%%%%%%%%%%%%%%%%%%%%%%%%%%%%%%%%%%%%%%%%%%%%%%%%%%%%%%%%%%%%%%%%
